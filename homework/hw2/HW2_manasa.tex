\documentclass[11pt]{article}
\usepackage{graphicx}
\graphicspath{ {C:/Users/Manasa/Pictures/} }

\begin{document}
\centerline{CS500 - Homework 2} 

\centerline{Manasa Navada} 

\centerline{manasa.navada@gmail.com} 
\null

\begin{flushleft}
\textbf{Exercise 7:} \\
\textit{Showthat if L is regular then $L^*$ is regular. Why does it not suffice to use the fact that the regular languages are closed under concatenation and union?}

\null
$L^*$ can be represented as: \\
$L^*$ = $\epsilon \cup L \cup LL \cup L^3 \cup L^4$...\\
The union of regular language is regular.\\
\null
If L1 is a regular language, there exists a regular expression r1 such that L1 = L(r1)\\
$r_1^*$ is a regular expression denoting $L_1^*$\\
This implies that $L_1^*$ is closed and it is regular. \\ 
\null
\textbf{Exercise 8:} \\
\textit{Given a string w, let $w^R$ denote w written in reverse. Given a language L, let $L^R = \{w^R | w \in L \}.$ Prove that L is regular if and only if $L^R$ is regular. Why is this harder to prove with DFAs?}\\
\null
Consider $L^R$ is regular.\\
To convert this into L, we can reverse all the strings of finite automata.\\
Make the old start state as the only new accepting state.\\
Create a new start state $p_0$ with $\delta(p_0, \epsilon)$ =F (The old accepting states)\\
The resulting NFA simulates L and therefore accepts a string w if and only if it accept $w^R$.\\
\null
This is harder to prove with DFA because we can not force a DFA to have a single final state. Second, a state may have two incoming transitions on the same character, resulting in non-determinism when reversed.\\
\null
\textbf{Exercise 9:} \\
\textit{A for-all NFA is one such that L(M) is the set of strings where every computation path ends in an accepting state. Show how to simulate an for-all NFA with a DFA, and thus prove that a language is recognized by some
for-all NFA if and only if it is regular.}\\
\null
Here we need to keep track of all possible states that an NFA can be in at each step. While the NFA’s transitions
are nondeterministic, this set changes in a deterministic way. Namely, after reading a symbol a, a state s is possible if s $\in \delta(t ,a)$ for some state t.\\
This lets us simulate an NFA M with a DFA $M_{DFA}$ as follows. \\
If $M's$ set of states is S, then $M_{DFA}'s$ set of states $S_{DFA}$ is the power set P(S)= $\{T \subseteq S\}$, its initial state $s_{DFA}^0$ is the set $\{s^0\}$, and its transition function is:\\
$\delta_{DFA}(T,a)= \bigcup_ {t \in T} \delta_M(t ,a)$ .\\
Then for any string w, $\delta_{DFA}^*(s_{DFA}^0,w)$ is the set of states that M could be in after reading w. Finally, M accepts if it could end in at least one accepting state, so we define the accepting set of $M_{DFA}$ as: \\
$S_{DFA}^{yes}$ = $\{T \cap S^{yes} \neq \phi\}$. \\
Then $L(M_{DFA})$= L(M).

\null
\textbf{Exercise 12:} \\
\textit{Given a language L, let $L_{1/2}$ denote the set of words that can appear as first halves of words in L:\\
\begin{center}
$L_{1/2} = \{x | \exists y : |x| = |y |$ and xy $\in$ L $\}$,
\end{center} 
where $|w|$ denotes the length of a word w. Prove that if L is regular, then $L_{1/2}$ is regular. Generalize this to $L_{1/3}$, the set of words that can appear as middle thirds of words in L:\\
\begin{center}
$L_{1/3} = \{ y | \exists x,z : |x| = |y |= |z |$ and xyz $\in L\}$
\end{center}
}.\\
start with a DFA that accepts L with states Q and accepting states F. The idea is that for xy to end in an accepting state, x must end in a “modified accepting state,” meaning a state which will end in an accepting state if we read y. Formally, we define the set of modified accepting states as $F_y$ = $\{q \in Q : \delta^*
(q, y) \in F\}$.\\
As we read x, we guess the symbols of y in reverse order. Specifically, when we read the $i^{th}$ symbol of x, we guess the $i^{th}$-to-last symbol b of y. Let y(i) denote the last i symbols of y; then we modify the accepting states from $F_{y(i-1)}$ to $F_{y(i)} = \{q \in Q : \delta(q, b) \in F_{y(i-1)}\}$.
After we have read all of x we have guessed $F_y$.
To turn this into an NFA for $L_{1/2}$, we just need to expand the set of states from Q to Q × P(Q) so we can keep track both of the DFA’s state and the modified $F_y \in P(Q)$. Then our accepting states are $\{(q, F_y) : q \in F_y\}$.\\
\null
Similarly we can generalize it for $L_{1/3}$ \\
\begin{center}
$L_{1/3} = \{ y | \exists x,z : |x| = |y |= |z |$ and xyz $\in L\}$
\end{center}
Then $L_{1/3}$ is the set of middle thirds of words in L, and by guessing the first and last thirds x and z we can prove that $L_{1/3}$ is regular if L is.\\
\null
\textbf{Exercise 13:} \\
\textit{Show that if u $\sim _L$ v , then ua $\sim_L$ va for any a $\in$ A.}\\
\null
Given that: u $\sim _L$ v. This means, for all w, uw $\in$ L if and only if vw $ \in$ L. This can contain words that begin with a (words that are of the form w = ab). So for all b, uab $ \in $ L if and only if vab $\in$ L hence we can say ua $\sim_L$ va.\\
\null
\textbf{Exercise 20:} \\
\textit{Consider the language: \\
\begin{center}
$L_{a=b,c=d}$ = $\{w \in \{a,b,c,d \}^*$  $|$  $\#_a(w)= \#_b(w)$ and $\#_c(w)=\#_d(w)\}$,
\end{center}
What are its equivalence classes? What does its minimal infinite-state machine look like?} \\
\null
For this language, there can be infinite possibilies of equivalence classes. The machine keeps track of inputs relative to other inputs. For example abcd is equivalent to aabbccdd.\\
Hence we can say that the language is not regular and it can not be accepted by the machine.\\
\null
\textbf{Exercise 14:} \\
\textit{Describe the equivalence classes of the three languages from Exercise 2. Use them to give the minimal
DFA for each language, or prove that the DFA you designed before is 
minimal.\\
Exercise 2: Part 1: The set of strings in $\{a,b\}^*$ with an even number of b’s.}\\
\null
The equilavence classes for this language is: \\
1. w contains zero b's (This is the start state)\\
2. w contains odd number of b's\\
3. w contains even number of b's(accepting state)\\
\null
Considering the following transition between these gives the following DFA:\\
\includegraphics[scale=0.5]{1}

Hence it is proved that the DFA we designed in HW1 is minimal.
\end{flushleft}

\end{document}