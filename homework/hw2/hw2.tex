%%%%%latex preamble%%%%%
\documentclass[titlepage]{article}\usepackage[]{graphicx}\usepackage[]{color}
%% maxwidth is the original width if it is less than linewidth
%% otherwise use linewidth (to make sure the graphics do not exceed the margin)
\makeatletter
\def\maxwidth{ %
  \ifdim\Gin@nat@width>\linewidth
  \linewidth
  \else
  \Gin@nat@width
  \fi
}
\makeatother

\usepackage{rotating}
\usepackage{listings}
\definecolor{mygreen}{rgb}{0,0.6,0}
\definecolor{mygray}{rgb}{0.5,0.5,0.5}
\definecolor{mymauve}{rgb}{0.58,0,0.82}
\lstset{ %
  backgroundcolor=\color{white},   % choose the background color; you must add \usepackage{color} or \usepackage{xcolor}
  basicstyle=\footnotesize,        % the size of the fonts that are used for the code
  breakatwhitespace=false,         % sets if automatic breaks should only happen at whitespace
  breaklines=true,                 % sets automatic line breaking
  captionpos=b,                    % sets the caption-position to bottom
  commentstyle=\color{mygreen},    % comment style
  deletekeywords={...},            % if you want to delete keywords from the given language
  escapeinside={\%*}{*)},          % if you want to add LaTeX within your code
  extendedchars=true,              % lets you use non-ASCII characters; for 8-bits encodings only, does not work with UTF-8
  frame=single,                    % adds a frame around the code
  keepspaces=true,                 % keeps spaces in text, useful for keeping indentation of code (possibly needs columns=flexible)
  keywordstyle=\color{blue},       % keyword style
  language=Python,                 % the language of the code
  morekeywords={*,...},            % if you want to add more keywords to the set
  numbers=left,                    % where to put the line-numbers; possible values are (none, left, right)
  numbersep=5pt,                   % how far the line-numbers are from the code
  numberstyle=\tiny\color{mygray}, % the style that is used for the line-numbers
  rulecolor=\color{black},         % if not set, the frame-color may be changed on line-breaks within not-black text (e.g. comments (green here))
  showspaces=false,                % show spaces everywhere adding particular underscores; it overrides 'showstringspaces'
  showstringspaces=false,          % underline spaces within strings only
  showtabs=false,                  % show tabs within strings adding particular underscores
  stepnumber=2,                    % the step between two line-numbers. If it's 1, each line will be numbered
  stringstyle=\color{mymauve},     % string literal style
  tabsize=2,                       % sets default tabsize to 2 spaces
  title=\lstname                   % show the filename of files included with \lstinputlisting; also try caption instead of title
}
\usepackage{alltt}
\usepackage[sc]{mathpazo}
\usepackage{amsmath, amsthm, amssymb}
\usepackage{graphicx}
\usepackage[T1]{fontenc}
\usepackage{geometry}
\geometry{verbose,tmargin=2.5cm,bmargin=2.5cm,lmargin=1.5cm,rmargin=1.5cm}
\setcounter{secnumdepth}{2}
\setcounter{tocdepth}{2}
\usepackage{url}
\usepackage[unicode=true,pdfusetitle,
  bookmarks=true,bookmarksnumbered=true,bookmarksopen=true,bookmarksopenlevel=2,
breaklinks=false,pdfborder={0 0 1},backref=false,colorlinks=false]
{hyperref}
\hypersetup{pdfstartview={XYZ null null 1}}
\usepackage{float}
\usepackage{bm}
\usepackage{tikz}
 %changes default sectioning commands -> 1,a, etc.
%\usepackage{breakurl}
\renewcommand{\thesubsection}{(\alph{subsection})}
\renewcommand{\thesubsubsection}{\roman{subsection}.}
\usepackage{lastpage}
\usepackage{fancyhdr}
\pagestyle{fancy}

%%% Header and Footer %%% 
\lhead{}
\chead{\leftmark}
\rhead{}
\lfoot{Aaron Gonzales; Theory of Computation}
\cfoot{Homework 2}
\rfoot{Page \thepage\ of \pageref{LastPage}}
\IfFileExists{upquote.sty}{\usepackage{upquote}}{}

\begin{document}

\title{More Fun With Automata \\ Homework 2, CS500, Fall 2014}
\author{Aaron Gonzales, \\ group 16}
\maketitle


%%%%%%%%%%%%%%%%%%%%%%%%%%%%%%%%%%%%%%%%%%%%%%%%
\section*{Ex 7}
\begin{quote}
  \textbf{Show that if $L$ is regular then $L^*$ is regular. Why does it not suffice
  to use the fact that the regular languages are closed under concatenation and
  union?}
\end{quote}
\subsubsection*{Answer:}
We know that $L* = \epsilon \cup L \cup LL \cup \dots$, the enumerated set of
all possible combinations of strings made up of the language $L$. We also know
that for a language to be regular, there must be some DFA or NFA that can
accept it. 

If we let $M_L$ be the machine that recognizes langague $L$, then let $M_{L^*}$
be the machine that recognizes $L^*$. To construct $M_{L^*}$, add a new start
state $S'$ that is also an accepting state to $M_L$. Connect $S'$ to $M_L$ with
an $\epsilon$ transition and add another $\epsilon$ edge out of $M_L$ such that
it connects to $S'$. This machine can accept any possible combinatins of valid
instances of $L$, showing that $L^*$ is indeed regular. 


%%%%%%%%%%%%%%%%%%%%%%%%%%%%%%%%%%%%%%%%%%%%%%%%

\section*{Ex 8}
\begin{quote}
  \textbf{Given a string $w$ , let $w^R$ denote $w$ written in reverse.
    Given a language $L$, let $L^R = \{w^R | w \in L \}$. Prove that $L$ is regular if and
    only if $L^R$ is regular. Why is this harder to prove with DFAs?}
\end{quote}
\subsubsection{Answer:}
We know that a language is regular if there is a DFA that can accept it. This
tells us that in order to prove $W^R$'s regularity, there must be some DFA that
can accept it. Generally put, we can map our current DFA to a DFA that can
accept the new reversed language by modifying the start state, accepting
states, and transition functions as follows, which will be formalized below.

\begin{proof} 
	Let $M = \{S, A, s^0, S^{yes}, \delta\}$ be the machine that recognizes $L$.
	Let $M' = {S',A',S^{0'}, S^{yes'}, \delta*'}$ be the NFA that recognizes
	$L^R$, constructed as below:

	Let $S_{new} \in S$ be a new start state and connect it to all $s \in S^{yes}$
	with an $\epsilon$ label. 
	Make $S^0$ into an $S^{yes'}$. 
	Make $s \in S^{yes} \to S'$ (make accepting states into normal states).
	Make $S^{0} \to S^{yes'}$ (make the original intial state into the sole
	accepting state.

	This new NFA will only accept langagues that begin in the first char of the
	$w^R$ and end in the last char of $w^r$. As we have constructed an NFA that
	recognizes this $L^R$, we can deduce that that $L^R$ is regular, meaning
	that $L \Leftrightarrow L^R$.
\end{proof}

The NFA epsilon ability allows us to more easily create new NFAs from
exisisting automata.


%%%%%%%%%%%%%%%%%%%%%%%%%%%%%%%%%%%%%%%%%%%%%%%%

\section*{Ex 12}
\begin{quote}
	\textbf{Given a language $L$, let $L_{1/2}$ denote the set of words that can appear
    as first halves of words in $L$:}
    \[ L_{1/2} = \{ x | \exists y : |x| = |y| \text{ and } x y \in L \} \]
    
  \textbf{where $|w |$ denotes the length of a word $w$ . Prove that if $L$ is
      regular, then $L_{1/2}$ is regular. Generalize this to $L_{1/3}$, the set of
      words that can appear as middle thirds of words in $L$:}
  \[ L_{1/3} = \{ y | \exists x,z : |x| = |y| = |z|\,  \text{and } x y z \in L \} \]
  
\end{quote}
\subsubsection*{Answer:}
We know that $|x| = |y|$ and must show that the set of words in $L_{1/2}$ can
be represented by a DFA to be proved regular. FA's limit us in that we cannot
go back in time or record with external memory, and we may not know how large a
word $w$ is before it is read. As such, we must (A) build a FA that allows us to
give $|x| = |y|$ and allows us to verify that (B) $xy \in L$. This is complicated
but not impossible.

Chris Moore's hint about River Song and the Doctor is a reference to Dr. Who,
in which the characters are moving toward each other in time, one going
forward and the other going backward. Given that hint, let us construct a FA
that allows us to accomplish A and B, mostly given by the product construction
of two FAs.

\begin{proof}
	Let us define two FAs:
	\begin{align}
		M =& \{Q, \Sigma, \delta, S_0, S^{yes}\} \\
		M^R =& \{ Q^R, \Sigma, \delta^R, S^{yes^R} \} \\
		Q^R = & Q \cup \{S_{0}^R \}
	\end{align}
	$M$ is the DFA which recognizses $L$ and $M^R$ is the DFA that recognizses
	$L^R$. We have seen that the reverse operator is closed under
	regularity. $Q^R$ is the union of states of $M$ and acepting states of
	$L^R$.

\end{proof}





%%%%%%%%%%%%%%%%%%%%%%%%%%%%%%%%%%%%%%%%%%%%%%%%

\section*{Ex 9}
\begin{quote}
  \textbf{A for-all NFA is one such that $L(M)$ is the set of strings where every
  computation path ends in an accepting state. Show how to simulate an for-all
  NFA with a DFA, and thus prove that a language is recognized by some for-all
  NFA if and only if it is regular.}
\end{quote}

\subsubsection*{Answer:}
\vspace{8cm}



%%%%%%%%%%%%%%%%%%%%%%%%%%%%%%%%%%%%%%%%%%%%%%%%
\section*{Ex 13}
\begin{quote}
  \textbf{Show that if $u ∼_L v$ , then $u a ∼_L v a$ for any $a \in A$.}
\end{quote}

\subsubsection*{Answer:}
\vspace{8cm}

blah blah \rotatebox{90}{\textasciitilde} blah blah blah
\[ W | \star \to \text{\rotatebox{90}{\textasciitilde}} \sum_{i}^x \]
%%%%%%%%%%%%%%%%%%%%%%%%%%%%%%%%%%%%%%%%%%%%%%%%
\section*{Ex 14}
\begin{quote}
  \textbf{Describe the equivalence classes of the three languages from Exercise
  2. Use them to give the minimal DFA for each language, or prove that the DFA
  you designed before is minimal.}
\end{quote}
\subsubsection{Answer:}
\vspace{8cm}




%%%%%%%%%%%%%%%%%%%%%%%%%%%%%%%%%%%%%%%%%%%%%%%%
\section*{Ex 20}
\begin{quote}
  \textbf{Consider the language}
  \[ L_{a=b,c=d } = \{ w \in \{a,b, c ,d \} \star ∗ \, | \#_a (w) = \#_b (w) \text{ and }
    \#_c (w) = \#_d (w )
  \]
  
\textbf{What are its equivalence classes? What does its minimal infinite-state machine
look like?}
\end{quote}

\subsubsection{Answer:}
\vspace{8cm}




\end{document}
