%%%%%latex preamble%%%%%
\documentclass[titlepage]{article}\usepackage[]{graphicx}\usepackage[]{color}
%% maxwidth is the original width if it is less than linewidth
%% otherwise use linewidth (to make sure the graphics do not exceed the margin)
\makeatletter
\def\maxwidth{ %
  \ifdim\Gin@nat@width>\linewidth
  \linewidth
  \else
  \Gin@nat@width
  \fi
}
\makeatother


\usepackage{listings}
\definecolor{mygreen}{rgb}{0,0.6,0}
\definecolor{mygray}{rgb}{0.5,0.5,0.5}
\definecolor{mymauve}{rgb}{0.58,0,0.82}
\lstset{ %
  backgroundcolor=\color{white},   % choose the background color; you must add \usepackage{color} or \usepackage{xcolor}
  basicstyle=\footnotesize,        % the size of the fonts that are used for the code
  breakatwhitespace=false,         % sets if automatic breaks should only happen at whitespace
  breaklines=true,                 % sets automatic line breaking
  captionpos=b,                    % sets the caption-position to bottom
  commentstyle=\color{mygreen},    % comment style
  deletekeywords={...},            % if you want to delete keywords from the given language
  escapeinside={\%*}{*)},          % if you want to add LaTeX within your code
  extendedchars=true,              % lets you use non-ASCII characters; for 8-bits encodings only, does not work with UTF-8
  frame=single,                    % adds a frame around the code
  keepspaces=true,                 % keeps spaces in text, useful for keeping indentation of code (possibly needs columns=flexible)
  keywordstyle=\color{blue},       % keyword style
  language=Python,                 % the language of the code
  morekeywords={*,...},            % if you want to add more keywords to the set
  numbers=left,                    % where to put the line-numbers; possible values are (none, left, right)
  numbersep=5pt,                   % how far the line-numbers are from the code
  numberstyle=\tiny\color{mygray}, % the style that is used for the line-numbers
  rulecolor=\color{black},         % if not set, the frame-color may be changed on line-breaks within not-black text (e.g. comments (green here))
  showspaces=false,                % show spaces everywhere adding particular underscores; it overrides 'showstringspaces'
  showstringspaces=false,          % underline spaces within strings only
  showtabs=false,                  % show tabs within strings adding particular underscores
  stepnumber=2,                    % the step between two line-numbers. If it's 1, each line will be numbered
  stringstyle=\color{mymauve},     % string literal style
  tabsize=2,                       % sets default tabsize to 2 spaces
  title=\lstname                   % show the filename of files included with \lstinputlisting; also try caption instead of title
}
\usepackage{alltt}
\usepackage[sc]{mathpazo}
\usepackage{amsmath, amsthm, amssymb}
\usepackage{graphicx}
\usepackage[T1]{fontenc}
\usepackage{geometry}
\geometry{verbose,tmargin=2.5cm,bmargin=2.5cm,lmargin=1.5cm,rmargin=1.5cm}
\setcounter{secnumdepth}{2}
\setcounter{tocdepth}{2}
\usepackage{url}
\usepackage[unicode=true,pdfusetitle,
  bookmarks=true,bookmarksnumbered=true,bookmarksopen=true,bookmarksopenlevel=2,
breaklinks=false,pdfborder={0 0 1},backref=false,colorlinks=false]
{hyperref}
\hypersetup{pdfstartview={XYZ null null 1}}
\usepackage{float}
\usepackage{bm}
\usepackage{tikz}
 %changes default sectioning commands -> 1,a, etc.
%\usepackage{breakurl}
\renewcommand{\thesubsection}{(\alph{subsection})}
\renewcommand{\thesubsubsection}{\roman{subsection}.}
\usepackage{lastpage}
\usepackage{fancyhdr}
\pagestyle{fancy}

%%% Header and Footer %%% 
\lhead{}
\chead{\leftmark}
\rhead{}
\lfoot{Aaron Gonzales; Theory of Computation}
\cfoot{Homework 2}
\rfoot{Page \thepage\ of \pageref{LastPage}}
\IfFileExists{upquote.sty}{\usepackage{upquote}}{}

\begin{document}

\title{More Fun With Automata \\ Homework 2, CS500, Fall 2014}
\author{Aaron Gonzales, \\ group 16}
\maketitle


%%%%%%%%%%%%%%%%%%%%%%%%%%%%%%%%%%%%%%%%%%%%%%%%
\section*{Ex 7}
\begin{quote}
  \textbf{Show that if $L$ is regular then $L^∗$ is regular. Why does it not suffice
  to use the fact that the regular languages are closed under concatenation and
  union?}
\end{quote}

\subsubsection*{Answer:}


%%%%%%%%%%%%%%%%%%%%%%%%%%%%%%%%%%%%%%%%%%%%%%%%

\section*{Ex 8}
\begin{quote}
  \textbf{Given a string $w$ , let $w^R$ denote $w$ written in reverse.
    Given a language $L$, let $L^R = \{w^R | w \in L \}$. Prove that $L$ is regular if and
    only if $L^R$ is regular. Why is this harder to prove with DFAs?}
\end{quote}

\begin{itemize}
  \item \textbf{}
  \item \textbf{ }
\end{itemize}


%%%%%%%%%%%%%%%%%%%%%%%%%%%%%%%%%%%%%%%%%%%%%%%%

\section*{Ex 12}
\begin{quote}
    \textbf{Given a language $L$, let $L1/2$ denote the set of words that can appear
    as first halves of words in $L$:}
    \[ L_{1/2} = \{ x | \exists y : |x| = |y| \text{ and } x y \in L \} \]
    
  \textbf{where $|w |$ denotes the length of a word $w$ . Prove that if $L$ is
      regular, then $L_{1/2}$ is regular. Generalize this to $L_{1/3}$, the set of
      words that can appear as middle thirds of words in $L$:}
  \[ L_{1/3} = \{ y | \exists x,z : |x| = |y| = |z|\,  \text{and } x y z \in L \} \]
  
\end{quote}
\subsubsection*{Answer:}



%%%%%%%%%%%%%%%%%%%%%%%%%%%%%%%%%%%%%%%%%%%%%%%%

\section*{Ex 9}
\begin{quote}
  \textbf{A for-all NFA is one such that $L(M)$ is the set of strings where every
  computation path ends in an accepting state. Show how to simulate an for-all
  NFA with a DFA, and thus prove that a language is recognized by some for-all
  NFA if and only if it is regular.}
\end{quote}

\subsubsection*{Answer:}
%%%%%%%%%%%%%%%%%%%%%%%%%%%%%%%%%%%%%%%%%%%%%%%%


\section*{Ex 13}
\begin{quote}
  \textbf{Show that if $u ∼_L v$ , then $u a ∼_L v a$ for any $a \in A$.}
\end{quote}

\subsubsection*{Answer:}



%%%%%%%%%%%%%%%%%%%%%%%%%%%%%%%%%%%%%%%%%%%%%%%%
\section*{Ex 14}
\begin{quote}
  \textbf{Describe the equivalence classes of the three languages from Exercise
  2. Use them to give the minimal DFA for each language, or prove that the DFA
  you designed before is minimal.}
\end{quote}




%%%%%%%%%%%%%%%%%%%%%%%%%%%%%%%%%%%%%%%%%%%%%%%%
\section*{Ex 20}
\begin{quote}
  \textbf{Consider the language}
  \[ L_{a=b,c=d } = \{ w \in \{a,b, c ,d \} \star ∗ \, | \#_a (w) = \#_b (w) \text{ and }
    \#_c (w) = \#_d (w )
  \]
  
\textbf{What are its equivalence classes? What does its minimal infinite-state machine
look like?}
\end{quote}

\subsubsection{Answer:}




\end{document}
