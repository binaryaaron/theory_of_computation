\documentclass[11pt]{article}
\usepackage{graphicx}
\graphicspath{ {C:/Users/Manasa/Pictures/} }
\begin{document}

\begin{flushleft}
\textbf{Exercise 14:} \\
\textit{Describe the equivalence classes of the three languages from Exercise 2. Use them to give the minimal
DFA for each language, or prove that the DFA you designed before is 
minimal.\\
Exercise 2: Part 3: The set of strings in $\{0,1\}^*$ that encode, in binary, an integer w that is a multiple of 3. Interpret the empty string $\epsilon$ as the number zero.}\\
\null
A minimal DFA for this language has three states. Each state represents the reminder of the number when it is divided by 3.\\
\null
The equivalence classes for this language are: \\
1. w containing the reminder 0. (Also the start state and accepting state)\\
2. w containing the reminder 1.\\
3. w containing the reminder 2.\\
\null
Considering the following transition between these gives the following DFA.\\
\includegraphics[scale=0.75]{3}\\
\null
Hence it is proved that the DFA we designed in HW1 is minimal. \\
\null
\end{flushleft}
\end{document}