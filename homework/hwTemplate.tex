%%%%%latex preamble%%%%%
\documentclass[titlepage]{article}
%% maxwidth is the original width if it is less than linewidth
%% otherwise use linewidth (to make sure the graphics do not exceed the margin)
\makeatletter
\def\maxwidth{ %
  \ifdim\Gin@nat@width>\linewidth
  \linewidth
  \else
  \Gin@nat@width
  \fi
}
\makeatother

\usepackage[]{graphicx}
\usepackage[]{color}

\usepackage{listings}
\usepackage{alltt}
% maths
\usepackage[sc]{mathpazo}
\usepackage{amsmath, amsthm, amssymb}
\usepackage{graphicx}
\usepackage[T1]{fontenc}
\usepackage{geometry}
\geometry{verbose,tmargin=2.5cm,bmargin=2.5cm,lmargin=1.5cm,rmargin=1.5cm}
\usepackage{url}
\usepackage{hyperref}
% removes border around hyperlinks
\hypersetup{pdfborder = {0 0 0}}
\usepackage{float}
\usepackage{bm}
\usepackage{tikz}

\setcounter{secnumdepth}{2}
\setcounter{tocdepth}{2}
 %changes default sectioning commands -> 1,a, etc.
\renewcommand{\thesubsection}{(\alph{subsection})}
\renewcommand{\thesubsubsection}{\roman{subsection}.}
\usepackage{lastpage}
\usepackage{fancyhdr}
\pagestyle{fancy}

%%% Header and Footer %%% 
\usepackage{lastpage}
\usepackage{fancyhdr}
%%% fancy pages for all non-title and chapter heading pages
\pagestyle{fancy}
%%% puts a line in the footer
\renewcommand{\footrulewidth}{0.4pt}
%%% adds a reference to the last page to give a running page count
\fancyfoot[R]{\thepage~of~\pageref{LastPage}} %requires lastpage
%%%% adds author name to bottom left
\fancyfoot[L]{\textit{Aaron Gonzales; Group 16}}
%%% empty center footer
\fancyfoot[C]{}

%%% Adds class name to left header
\fancyhead[L]{\textit{Theory of Computation}}
%%% Adds document name to right header
\fancyhead[R]{\textit{Homework X}}

\IfFileExists{upquote.sty}{\usepackage{upquote}}{}

\begin{document}

\title{Some title\\ Homework X, CS500, Fall 2014}
\author{Aaron Gonzales, \\ group 16}
\maketitle


%%%% useful align for this
\section{Template math}
\begin{align*}
	Total Matches &={n \choose 3} * \frac{1}{25} \\
					&= {27 \choose 3}   * \frac{1}{25} \\
					&=   \frac{27!}{3!(27-3)!} * \frac{1}{25} \\
					&= \frac{27 * 26 * 25}{6} * \frac{1}{25}  \\
					&= 117
\end{align*}


We have $n$ debutants and $n$ porsches. 
$\frac{1}{n}$ is our probability of a debutant getting into her own porche.
Let $X_i$ be an indicator variable saying 

\[
	X_i = I[X]
	\begin{cases}
		1 & \text{if debutante gets in her own car} \\
		0 & \text{if not} 
	\end{cases}
\]

The number of debutants who return to their own can be expressed as:
\[ X_i = \sum_{i=1}^n X_i \]

and the expected number of debutants can be expressed as:

\[ E [X_i] = E \left[ \sum_{i=1}^n X_i \right] \]

By linearity of expectation we can state it as such:

\begin{align*}
	E [X_i] &= E \left[ \sum_{i=1}^n X_i \right] \\
	E [X_i] &=  \sum_{i=1}^n E[X_i] \\ 
	&=  \sum_{i=1}^n \frac{1}{n} \\ 
	&=  \frac{1}{n} \sum_{i=1}^n 1 \\
	&= \frac{n}{n} \\
	& = 1
\end{align*}

We only expect one drunken debutant to get in her own porche. (Note --- This presumes
that all of the debutants made it out of the party and no one passed out
inside.)


\section{}
\begin{quote}
  \textbf{}
\end{quote}

\begin{itemize}
  \item \textbf{}
  \item \textbf{}
\end{itemize}











\end{document}
