%%%%%latex preamble%%%%%
\documentclass[titlepage]{article}\usepackage[]{graphicx}\usepackage[]{color}
%% maxwidth is the original width if it is less than linewidth
%% otherwise use linewidth (to make sure the graphics do not exceed the margin)
\makeatletter
\def\maxwidth{ %
  \ifdim\Gin@nat@width>\linewidth
  \linewidth
  \else
  \Gin@nat@width
  \fi
}
\makeatother


\usepackage{listings}
\definecolor{mygreen}{rgb}{0,0.6,0}
\definecolor{mygray}{rgb}{0.5,0.5,0.5}
\definecolor{mymauve}{rgb}{0.58,0,0.82}
\lstset{ %
  backgroundcolor=\color{white},   % choose the background color; you must add \usepackage{color} or \usepackage{xcolor}
  basicstyle=\footnotesize,        % the size of the fonts that are used for the code
  breakatwhitespace=false,         % sets if automatic breaks should only happen at whitespace
  breaklines=true,                 % sets automatic line breaking
  captionpos=b,                    % sets the caption-position to bottom
  commentstyle=\color{mygreen},    % comment style
  deletekeywords={...},            % if you want to delete keywords from the given language
  escapeinside={\%*}{*)},          % if you want to add LaTeX within your code
  extendedchars=true,              % lets you use non-ASCII characters; for 8-bits encodings only, does not work with UTF-8
  frame=single,                    % adds a frame around the code
  keepspaces=true,                 % keeps spaces in text, useful for keeping indentation of code (possibly needs columns=flexible)
  keywordstyle=\color{blue},       % keyword style
  language=Python,                 % the language of the code
  morekeywords={*,...},            % if you want to add more keywords to the set
  numbers=left,                    % where to put the line-numbers; possible values are (none, left, right)
  numbersep=5pt,                   % how far the line-numbers are from the code
  numberstyle=\tiny\color{mygray}, % the style that is used for the line-numbers
  rulecolor=\color{black},         % if not set, the frame-color may be changed on line-breaks within not-black text (e.g. comments (green here))
  showspaces=false,                % show spaces everywhere adding particular underscores; it overrides 'showstringspaces'
  showstringspaces=false,          % underline spaces within strings only
  showtabs=false,                  % show tabs within strings adding particular underscores
  stepnumber=2,                    % the step between two line-numbers. If it's 1, each line will be numbered
  stringstyle=\color{mymauve},     % string literal style
  tabsize=2,                       % sets default tabsize to 2 spaces
  title=\lstname                   % show the filename of files included with \lstinputlisting; also try caption instead of title
}
\usepackage{alltt}
\usepackage[sc]{mathpazo}
\usepackage{amsmath, amsthm, amssymb}
\usepackage{graphicx}
\usepackage[T1]{fontenc}
\usepackage{geometry}
\geometry{verbose,tmargin=2.5cm,bmargin=2.5cm,lmargin=1.5cm,rmargin=1.5cm}
\setcounter{secnumdepth}{2}
\setcounter{tocdepth}{2}
\usepackage{url}
\usepackage[unicode=true,pdfusetitle,
  bookmarks=true,bookmarksnumbered=true,bookmarksopen=true,bookmarksopenlevel=2,
breaklinks=false,pdfborder={0 0 1},backref=false,colorlinks=false]
{hyperref}
\hypersetup{pdfstartview={XYZ null null 1}}
\usepackage{float}
\usepackage{bm}
\usepackage{tikz}
 %changes default sectioning commands -> 1,a, etc.
%\usepackage{breakurl}
\renewcommand{\thesubsection}{(\alph{subsection})}
\renewcommand{\thesubsubsection}{\roman{subsection}.}
\usepackage{lastpage}
\usepackage{fancyhdr}
\pagestyle{fancy}

%%% Header and Footer %%% 
\lhead{}
\chead{\leftmark}
\rhead{}
\lfoot{Aaron Gonzales; Programming Languages}
\cfoot{Homework 1}
\rfoot{Page \thepage\ of \pageref{LastPage}}
\IfFileExists{upquote.sty}{\usepackage{upquote}}{}

\begin{document}

\title{Homework 1, CS550, Spring 2015}
\author{Aaron Gonzales}
\maketitle


\section{}
\begin{quote}
  \textbf{Let p,q,r be proposistions:}
\end{quote}

\begin{itemize}
  \item \textbf{p: You have the flu}
  \item \textbf{q: You miss the final exam}
  \item \textbf{r: You pass the course}
\end{itemize}

\subsection{ $(p \implies q)$ }
\subsubsection{Answer:}
``I have the flu and as such, I missed the final''.

\subsection{ $(q \implies \lnot q)$ }
\subsubsection{Answer:}
``I missed the final and as such, did not pass the course.''

\subsection{ $[(p \implies \lnot r) \lor (q \implies \lnot r)]$ }
\subsubsection{Answer:}
``I had the flu and as such I did not pass the course or I missed the final exam
and did not pass the course.''

\subsection{ $[(p \land q) \lor ( \lnot q \land r)]$ }
\subsubsection{Answer:}
``I missed the final and had flu or I didn't miss the final and i passed the
course.''


\section{ }
\begin{quote}
  \textbf{Let p,q,r be proposistions:}
\end{quote}

\begin{itemize}
  \item \textbf{p: you get an A on the final exam}
  \item \textbf{q: You do every exercise in the book}
  \item \textbf{r: You get an A in this class}
\end{itemize}
\subsection{You get an A on the final but you do not do every exercise in the
book; nevertheless, you get an aA in this class}
\subsubsection{Answer:}
\( (p \land \lnot q) \implies r \)

\subsection{Getting an A on the final and doing every exercise in teh book is
sufficient for getting an A in the class}
\subsubsection{Answer:}
\( ( p \land q) \implies r \)

\subsection{You will get an A in the class if and only if you either do every
exercise in the book or you get an A on the final}
\subsubsection{Answer:}
$ ( p \lor q) \implies r $


\section{ }
\begin{quote}
  \textbf{Determine whether the argument is correct or incorrect and explain
  why}
\end{quote}
\subsection{Everyone enrolled in the university has lived in a dormitory. Mia
has never lived in a dormitory. Therefore, Mia is not enrolled in the
university.}
\subsubsection{Answer:}
True. All students in the university have lived in dormitories and Mia has not.
If she was enrolled in the university, she would live or have lived in a dorm.  

\subsection{A convertible car is fun to drive. Isaac's car is not convertible.
Therefore, Issac's car is not fun to drive.}
\subsubsection{Answer:}
False. The predicate makes no mention about all cars level of fun to drive,
only convertible cars. Other cars may be fun to drive as well.

\subsection{Quincy likes all action movies. Quincy likes the movie
  \textit{Eight Men Out.} Therefore, \textit{Eight Men Out} is an action
movie.}
\subsubsection{Answer:}
False. The predicate makes no mention if Quincy likes other types of movies and
only establishes something for his love of action movies. \textit{Eight Men
Out} may be another type of movie that he likes.

\subsection{All lobstermen set at least a dozen traps. Hamilton is a
lobsterman. Therefore, Hamilton sets at least a dozen traps.}
\subsubsection{Answer:}
True. A basic fact of being a lobsterman is setting a dozen traps. If Hamilton
violates this, he is not a lobsterman. 



%%%%%%%%%%%%%%%%
\section{}
\begin{quote}
  \textbf{Express each of these sytem specifications using predicates,
  quantifies, and logical connectives:}
\end{quote}
  \subsection{ Every user has access to an electronic mailbox }
  \subsubsection{Answer:}
  Let a user be denoted by $x$ and access to an electronic mailbox be $P(x)$.
  \[ \forall x P(x) \]


  \subsection{ The system mailbox can be accessed by everyone in the group if
  the file system is locked. }
  \subsubsection{Answer:}
  Let $x$ be a user \textit{in the group} and $P(x)$ denote access to the system.
  \[ \forall x P(x) \]

  \subsection{ The firewall is in a diagnostic state only if the proxy server
  is in  a diagnostic state. }
  \subsubsection{Answer:}
  let $D(x)$ be the proxy server's diagnostic state and $F(x)$ be the
  firewall's diagnostic state.
  \[ D(x) \implies F(x) \]

  \subsection{ At least one router is functioning normally if the throughput is
  between 100kpbs and 500 kbps and the proxy server is not in diagnostic
  mode.}
  \subsubsection{Answer:}
  let $R(r)$ be true if a router is functioning normally and $T(x)$ be throughput of 100 to 500 kbps. 
  \[ (T(x) \land \lnot D(x)) \implies R(r) \]


\section{ }
  \subsection{Which of these compound propositions are satisfiable, falsifiable,
  both satisfiable and falsifiable?}
    \begin{itemize}
      \item \( ( p \lor q \lor \lnot r) \land ( p \lor \lnot q \lor \lnot s) \)
    \end{itemize}


  \subsection{ Prove that $ p \implies [q \implies (p \land q)] $ and $ (p
      \land q) \implies q \implies (p \lor r) $ are tautologies or not. Is $( p
      \lor q) \implies [q \implies q] $ a tautology? } 
  \subsubsection{Answer}

  \subsection{ }
  \begin{itemize}
      \item Show that $\lnot p \implies (q \implies r) $ and 
        $ q \implies (p \lor r)$ are logically equivalent.
      \item Show that $( p \land q) \implies r$ and $(p \implies r) \land (q
        \implies r)$ are not equivalent.
    \end{itemize}


  \subsubsection{Answer}

  \subsection{ Show that $ ( p \implies q ) \land ( p \implies r ) $ and 
  $ p \implies (p \land r))$ are logically equivalent.}
  \subsubsection{Answer}


  %%%%%%%%%%%%%%%%%%%%%%%%%%%%
  \section{}
  \subsection{Write a formula that expresses that a number $g$ is the GCD of
  two other numbers $x,y$}

  \subsection{}

  \section{}
  \subsection{What is the distinction between the following two formulas?}
  \[ \forall x \, \exists y \, P(x,y). \]
  \[ \exists x \, \forall y \, P(x,y). \]
  \subsubsection{Answer:}



  \subsection{Does $(\forall x \, P(x)) \lor (\forall x \, Q(x))$ imply $\forall
    x (P(x) \lor Q(x))$. What about the other way? if it does not imply, give a
    coutnerexample illustrating it. If it does imply, how would you convince
    someone about it?}
    \subsubsection{Answer:}

\section{ }
  \begin{quote}
    {In a logic programming language, define relations (i.e., write programs)
    to determine if a list:}
  \end{quote}

  \subsection{is a permutation of another list}
  \subsubsection{Answer:}

  \subsection{has an even number of elements in it}
  \subsubsection{Answer:}

  \subsection{is formed by merging two other lists, i.e., successively
  taking one element at at ime from each list and merging until one of the
lists becomes empty, in which take the remaining nonempy list.}
  \subsubsection{Answer:}


  \end{document}
