%%%%%latex preamble%%%%%
\documentclass[titlepage]{article}\usepackage[]{graphicx}\usepackage[]{color}
%% maxwidth is the original width if it is less than linewidth
%% otherwise use linewidth (to make sure the graphics do not exceed the margin)
\makeatletter
\def\maxwidth{ %
  \ifdim\Gin@nat@width>\linewidth
  \linewidth
  \else
  \Gin@nat@width
  \fi
}
\makeatother

\usepackage{rotating}
\usepackage{listings}
\definecolor{mygreen}{rgb}{0,0.6,0}
\definecolor{mygray}{rgb}{0.5,0.5,0.5}
\definecolor{mymauve}{rgb}{0.58,0,0.82}
\lstset{ %
  backgroundcolor=\color{white},   % choose the background color; you must add \usepackage{color} or \usepackage{xcolor}
  basicstyle=\footnotesize,        % the size of the fonts that are used for the code
  breakatwhitespace=false,         % sets if automatic breaks should only happen at whitespace
  breaklines=true,                 % sets automatic line breaking
  captionpos=b,                    % sets the caption-position to bottom
  commentstyle=\color{mygreen},    % comment style
  deletekeywords={...},            % if you want to delete keywords from the given language
  escapeinside={\%*}{*)},          % if you want to add LaTeX within your code
  extendedchars=true,              % lets you use non-ASCII characters; for 8-bits encodings only, does not work with UTF-8
  frame=single,                    % adds a frame around the code
  keepspaces=true,                 % keeps spaces in text, useful for keeping indentation of code (possibly needs columns=flexible)
  keywordstyle=\color{blue},       % keyword style
  language=Python,                 % the language of the code
  morekeywords={*,...},            % if you want to add more keywords to the set
  numbers=left,                    % where to put the line-numbers; possible values are (none, left, right)
  numbersep=5pt,                   % how far the line-numbers are from the code
  numberstyle=\tiny\color{mygray}, % the style that is used for the line-numbers
  rulecolor=\color{black},         % if not set, the frame-color may be changed on line-breaks within not-black text (e.g. comments (green here))
  showspaces=false,                % show spaces everywhere adding particular underscores; it overrides 'showstringspaces'
  showstringspaces=false,          % underline spaces within strings only
  showtabs=false,                  % show tabs within strings adding particular underscores
  stepnumber=2,                    % the step between two line-numbers. If it's 1, each line will be numbered
  stringstyle=\color{mymauve},     % string literal style
  tabsize=2,                       % sets default tabsize to 2 spaces
  title=\lstname                   % show the filename of files included with \lstinputlisting; also try caption instead of title
}
\usepackage{alltt}
\usepackage[sc]{mathpazo}
\usepackage{amsmath, amsthm, amssymb}
\usepackage{graphicx}
\usepackage[T1]{fontenc}
\usepackage{geometry}
\geometry{verbose,tmargin=2.5cm,bmargin=2.5cm,lmargin=1.5cm,rmargin=1.5cm}
\setcounter{secnumdepth}{2}
\setcounter{tocdepth}{2}
\usepackage{float}
\usepackage{bm}
\usepackage{tikz}
\usetikzlibrary{arrows,automata}

 %changes default sectioning commands -> 1,a, etc.
%\usepackage{breakurl}
\renewcommand{\thesubsection}{(\alph{subsection})}
\renewcommand{\thesubsubsection}{\roman{subsection}.}

\usepackage{url}
\usepackage{hyperref}
\hypersetup{pdfborder = {0 0 0}}

%%% Header and Footer %%% 
\usepackage{lastpage}
\usepackage{fancyhdr}
%%% fancy pages for all non-title and chapter heading pages
\pagestyle{fancy}
%%% puts a line in the footer
\renewcommand{\footrulewidth}{0.4pt}
%%% adds a reference to the last page to give a running page count
\fancyfoot[R]{\thepage of \pageref{LastPage}} %requires lastpage
%%%% adds author name to bottom left
\fancyfoot[L]{\textit{Aaron Gonzales}}
%%% empty center footer
\fancyfoot[C]{}

%%% Adds class name to left header
\fancyhead[L]{\textit{Theory of Computation}}
%%% Adds document name to right header
\fancyhead[R]{\textit{Homework 3}}

\IfFileExists{upquote.sty}{\usepackage{upquote}}{}

\begin{document}

\title{More Fun With Automata \\ Homework 3, CS500, Fall 2014}
\author{Aaron Gonzales (group 16)}
\maketitle


%%%%%%%%%%%%%%%%%%%%%%%%%%%%%%%%%%%%%%%%%%%%%%%%
\section{Ex 23, Automata Notes}
\begin{quote}
  \textbf{Given a language \(L\), the language \(sort(L)\) consists of the words
  in L with their characters sorted in alphabetical order. For instance, if }
  \[ L = \{ bab, cca, abc\} \]
  then
  \[ sort(L) = \{ abb, acc, abc\}. \]
  \textbf{ Give an example of a regular langauge \(L_1\) such that
  \(sort(L_1)\)  is nonregular and a nonregular language \(sort(L_2)\) such
  that \(sort(L_2)\) is regular. You may use any technique you like to prove that
  the languages are regular.}
\end{quote}
\subsubsection*{Answer:}
\vspace{5cm}

%%%%%%%%%%%%%%%%%%%%%%%%%%%%%%%%%%%%%%%%%%%%%%%%

%%%%%%%%%%%%%%%%%%%%%%%%%%%%%%%%%%%%%%%%%%%%%%%%
\section{Infinite sequences of languages}
\begin{quote}
  \textbf{Find an infinite sequence of languages \( A_0 \subset A_1 \subset A_2
    \subset \dots \subset A_k \subset \dots \)
    such that for each even \( n \), \( A_n\) is regular, and for each odd
    \(n\), \(A_n \) is non-regular. Prove your solution is correct.}
\end{quote}
\subsubsection*{Answer:}
\vspace{5cm}

%%%%%%%%%%%%%%%%%%%%%%%%%%%%%%%%%%%%%%%%%%%%%%%%

%%%%%%%%%%%%%%%%%%%%%%%%%%%%%%%%%%%%%%%%%%%%%%%%
\section{Regex Golf}
\begin{quote}
  \textbf{Go to \href{https://regex.alf.nu/} and solve at least 5 of the
  puzzles.  Solving means finding a regular expression that matches a substring
of every string on the "match" list, and no substring of any string on the
"none of these" list. Of your solutions, submit the 5 you like best, along
with the score for each.  Your solutions should be proper regular expressions,
defined as follows:}
\begin{itemize}
  \item You may use ranges, such as [a-z]
  \item You may use the start-of-string character: \^
    and the end-of-string character: \$.
  \item You may use the OR character: |; 
    the Kleene star operator: *; and parentheses: (, ).
  \item You may NOT use backrefs or other constructs that allow the construction of
        expressions that match non-regular languages.  (The server allows some of
        these, despite calling the game "regexp golf," but this assignment does not.)
\end{itemize}
\end{quote}
\subsubsection*{Answer:}
\vspace{5cm}

%%%%%%%%%%%%%%%%%%%%%%%%%%%%%%%%%%%%%%%%%%%%%%%%

%%%%%%%%%%%%%%%%%%%%%%%%%%%%%%%%%%%%%%%%%%%%%%%%
\section{Context-Free Grammars}
\begin{quote}
  \textbf{Give Context-Free Grammars that generate the following languages over
    alphabet \(\{0,1\}\). Also say whether each language is regular.}
\end{quote}
\subsection{\(\{w : w \ \mbox{contains at least two 1's}\}\)}
\subsubsection*{Answer:}
\vspace{5cm}


\subsection{\(\{w : w \ \mbox{starts and ends with the same symbol, and has odd length}\}\)}
\subsubsection*{Answer:}
\vspace{5cm}



\subsection{\(\{wx : x \ \mbox{is a substring of the reverse of} \ w\}\)}
\subsubsection*{Answer:}
\vspace{5cm}
%%%%%%%%%%%%%%%%%%%%%%%%%%%%%%%%%%%%%%%%%%%%%%%%

%%%%%%%%%%%%%%%%%%%%%%%%%%%%%%%%%%%%%%%%%%%%%%%%
\section{Grammar and language}
\begin{quote}
  \textbf{What language is generated by the following grammar?
    Prove whether it is a regular language or not. THere are 3 variables:
  \(S, A, B\) and two terminals \(\{0,1\}\) }
\end{quote}
\subsection{\(S \to AA,B\)}
\subsubsection*{Answer:}
\vspace{5cm}


\subsection{\(A \to 0A, A0, 1\)}
\subsubsection*{Answer:}
\vspace{5cm}


\subsection{\(B \to 0B00,1\)}
\subsubsection*{Answer:}
\vspace{5cm}

\section{Exercise 36, Automata notes}
\begin{quote}
  \textbf{Show that a 1-DCA can be simulated by a DPDA, and similarly for
  1-NCAs and NPDAs. Do you think this is true for two-counter automata as well?}
\end{quote}
\subsubsection{Answer:}
\vspace{5cm}



\end{document}
