%%%%%latex preamble%%%%%
\documentclass[titlepage]{article}\usepackage[]{graphicx}\usepackage[]{color}
%% maxwidth is the original width if it is less than linewidth
%% otherwise use linewidth (to make sure the graphics do not exceed the margin)
\makeatletter
\def\maxwidth{ %
  \ifdim\Gin@nat@width>\linewidth
  \linewidth
  \else
  \Gin@nat@width
  \fi
}
\makeatother

\usepackage{rotating}
\usepackage{listings}
\usepackage{alltt}
\usepackage[sc]{mathpazo}
\usepackage{amsmath, amsthm, amssymb}
\usepackage{graphicx}
\usepackage[T1]{fontenc}
\usepackage{geometry}
\geometry{verbose,tmargin=2.5cm,bmargin=2.5cm,lmargin=1.5cm,rmargin=1.5cm}
\setcounter{secnumdepth}{2}
\setcounter{tocdepth}{2}
\usepackage{float}
\usepackage{bm}
\usepackage{tikz}
\usetikzlibrary{arrows,automata}

 %changes default sectioning commands -> 1,a, etc.
%\usepackage{breakurl}
\renewcommand{\thesubsection}{(\alph{subsection})}
\renewcommand{\thesubsubsection}{\roman{subsection}.}

\usepackage{url}
\usepackage{hyperref}
\hypersetup{pdfborder = {0 0 0}}

%%% Header and Footer %%% 
\usepackage{lastpage}
\usepackage{fancyhdr}
%%% fancy pages for all non-title and chapter heading pages
\pagestyle{fancy}
%%% puts a line in the footer
\renewcommand{\footrulewidth}{0.4pt}
%%% adds a reference to the last page to give a running page count
\fancyfoot[R]{\thepage of \pageref{LastPage}} %requires lastpage
%%%% adds author name to bottom left
\fancyfoot[L]{\textit{Aaron Gonzales}}
%%% empty center footer
\fancyfoot[C]{}

%%% Adds class name to left header
\fancyhead[L]{\textit{Theory of Computation}}
%%% Adds document name to right header
\fancyhead[R]{\textit{Homework 3}}

\IfFileExists{upquote.sty}{\usepackage{upquote}}{}

\begin{document}

\title{Even More Fun With Automata \\ Homework 3, CS500, Fall 2014}
\author{Aaron Gonzales (group 16)}
\maketitle

Note to teammates: I'm sorry about the low quality of this assignment. I will
make myself available as much as possible to correct and compile our solution
set and will continue to work on these before we meet.

%%%%%%%%%%%%%%%%%%%%%%%%%%%%%%%%%%%%%%%%%%%%%%%%
\section{Ex 23, Automata Notes}
\begin{quote}
  \textbf{Given a language \(L\), the language \(sort(L)\) consists of the words
  in L with their characters sorted in alphabetical order. For instance, if }
  \[ L = \{ bab, cca, abc\} \]
  then
  \[ sort(L) = \{ abb, acc, abc\}. \]
  \textbf{Give an example of a regular langauge \(L_1\) such that
  \(sort(L_1)\)  is nonregular and a nonregular language \(sort(L_2)\) such
  that \(sort(L_2)\) is regular. You may use any technique you like to prove that
  the languages are regular.}
\end{quote}
\subsection{Regular to Nonregular}
Let \[ L_1 = \text{palindrome over } \{ a,b \} \]
This implies that 
\[ sort(L_1) = \{a^mb^n | \text{either m or n is odd, not both}, |m|, |n|, \geq
0\} \]
as an example, the palindrome $abababa$ when sorted becomes $aaaabbb$. 

\subsection{Nonregular to regular}
Let \[ L_2 = (abaa^*)^* \].
Sorting $L_2$ gives us \[ a^mb^n | m > n \geq 0 \].

%%%%%%%%%%%%%%%%%%%%%%%%%%%%%%%%%%%%%%%%%%%%%%%%

%%%%%%%%%%%%%%%%%%%%%%%%%%%%%%%%%%%%%%%%%%%%%%%%
\section{Infinite sequences of languages}
\begin{quote}
  \textbf{Find an infinite sequence of languages \( A_0 \subset A_1 \subset A_2
    \subset \dots \subset A_k \subset \dots \)
    such that for each even \( n \), \( A_n\) is regular, and for each odd
    \(n\), \(A_n \) is non-regular. Prove your solution is correct.}
\end{quote}
\subsubsection*{Answer:}

This was a difficult problem. 

Taking our example from problem 1, define two languages:
Let below be the language when $n$ is odd
\begin{align*}
L_n =\{ \text{palindrome over } \{ a^i,b^{i-1} \} \} \\
\end{align*}
and below bet the language when $n$ is even. 
\begin{align*}
sort(L_n) = \{a^mb^p | \text{either m or p is odd, not both}, |m|, |p|, \geq 0\} 
\end{align*}
Sorting the palindromic language gives us regularity and the next step takes us out of it.




\section{Regex Golf}
\begin{quote}
  \textbf{Go to \href{https://regex.alf.nu/} and solve at least 5 of the
  puzzles.  Solving means finding a regular expression that matches a substring
of every string on the "match" list, and no substring of any string on the
"none of these" list. Of your solutions, submit the 5 you like best, along
with the score for each.  Your solutions should be proper regular expressions,
defined as follows:}
\begin{itemize}
  \item You may use ranges, such as [a-z]
  \item You may use the start-of-string character: \^
    and the end-of-string character: \$.
  \item You may use the OR character: |; 
    the Kleene star operator: *; and parentheses: (, ).
  \item You may NOT use backrefs or other constructs that allow the construction of
        expressions that match non-regular languages.  (The server allows some of
        these, despite calling the game "regexp golf," but this assignment does not.)
\end{itemize}
\end{quote}
\subsubsection*{Answer:}
\begin{itemize}
  \item level: warmup; \( foo \) : 207 pts
  \item level: anchors; \( k\$ \) : 208 pts
  \item level: ranges; \begin{verbatim}/^[a-f].*[a-f][abd-f]$/ \end{verbatim}: 179 pts
  \item level: long count; \begin{verbatim}/ 0.*1 0010 0011 0100 0101 0110 01+ 10+ 1001 1010 1011 1100 1101 1+0 1+$ / \end{verbatim} : 200 pts
  \item level: glob; \begin{verbatim}/^((er|f|i|p|t|v)|b|c[^a]).*[b-jm-z]$/\end{verbatim}: 124 pts but i failed matching many cases.
\end{itemize}

%%%%%%%%%%%%%%%%%%%%%%%%%%%%%%%%%%%%%%%%%%%%%%%%

%%%%%%%%%%%%%%%%%%%%%%%%%%%%%%%%%%%%%%%%%%%%%%%%
\section{Context-Free Grammars}
\begin{quote}
  \textbf{Give Context-Free Grammars that generate the following languages over
    alphabet \(\{0,1\}\). Also say whether each language is regular.}
\end{quote}
\subsection{\(\{w : w \ \mbox{contains at least two 1's}\}\)}
\subsubsection*{Answer:}
\begin{align*}
S \to 1A, 0A \\
A \to 1A, 0A, B \\
B \to 1, A, C
\end{align*}
\subsection{\(\{w : w \ \mbox{starts and ends with the same symbol, and has odd length}\}\)}
\subsubsection*{Answer:}

\begin{align*}
  G = {V, \Sigma, \Gamma, S} & \\
  V = & \{S, A\} \\
  \Sigma = & \{0,1\} \\
  S = & S \\
  S \to & A | 1 | 0 \\
  A \to & 0B0 | 1B1 \\
  B \to & 0B0 | 1B1 | 1B0 | 0B1 | 0 | 1
\end{align*} 
The first rule accounts for the singleton case, e.g. a string of 1
or 0. $A$ ensures that the word starts and ends with either 1 or 0. $B$ allows
only odd filler values. 

\subsection{\(\{wx : x \ \mbox{is a substring of the reverse of} \ w\}\)}
\subsubsection*{Answer:}
Unanswered.



%%%%%%%%%%%%%%%%%%%%%%%%%%%%%%%%%%%%%%%%%%%%%%%%

%%%%%%%%%%%%%%%%%%%%%%%%%%%%%%%%%%%%%%%%%%%%%%%%
\section{Grammar and language}
\begin{quote}
  \textbf{What language is generated by the following grammar?
    Prove whether it is a regular language or not. THere are 3 variables:
  \(S, A, B\) and two terminals \(\{0,1\}\) }
\end{quote}
\begin{align*}
  S \to& AA,B \\
  A \to& 0A, A0, 1 \\
  B \to& 0B00,1
\end{align*}
\subsubsection{Answer}

\[ L = \left( \left( 0^* \right)1\left( 0^* \right)1( 0^* \right) | 0^n 1 0^{2n} \]

No proof. I'll leave it as an exercise for me to do after the assignment is
due.

\section{Exercise 36, Automata notes}
\begin{quote}
  \textbf{Show that a 1-DCA can be simulated by a DPDA, and similarly for
  1-NCAs and NPDAs. Do you think this is true for two-counter automata as well?}
\end{quote}
\subsubsection{Answer:}

For DPDA $\to$ 1--DCA The stack symbol can be thought of as an increment or
decrement operation, allowing us to keep a pseudo "count" of the number of
things that happen.

For NPDA $\to$ 1--NCA, the same method as above follows.

For the two counter automata, they can recognize a language that a DPDA cannot,
namely

\[ L = a^p b^p c^p | p > 1\]

The proof is left as an exercise for the reader... who is me.

\end{document}
